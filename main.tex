\documentclass[12pt]{article}

\usepackage{graphicx} % Required for inserting images
\graphicspath{{images/}} %Direct to "images" folder
\usepackage{conveniences, geometry, ragged2e, amssymb, titlesec, fancyhdr, xcolor, etoolbox, float}
\usepackage[titles]{tocloft}  % Allows customization of ToC layout
\usepackage[nottoc]{tocbibind}

\setlength{\headheight}{14.5pt}

\usepackage[style=ieee, sorting=ynt]{biblatex}
\addbibresource{references.bib}

\usepackage[hidelinks]{hyperref}

\titleformat{\section}
    {\centering\fontS{18}\bfseries} %Centered, Font 18, Bold
    {\thesection} %No numbering
    {0.5em} %No extra space
    {} %No extra formatting
    [\vspace{20pt}\titlerule\vspace{10pt}]

\begin{document}
\begin{center}
    \includegraphics[width=\linewidth]{NTU_Logo.png}
    \\[1cm]
    \fontS{20}
    \underline{\textbf{CY2001 Research Attachment 1}}
    \\[2cm]
    \fontS{16}
    \textbf{Picture-Hanging Puzzles %Probably narrow down later
    \\[4em]
    \fontS{14}
    SARAH EMILY ONG XIN WEI
    \\[0.5cm]
    U2440124G
    \\[0.5cm]
    Mathematical Sciences
    \\[0.5cm]
    School of Physical and Mathematical Sciences
    \\[2cm]
    Academic Year 2025/2026
    \\[0.5cm]
    Second Semester}
\end{center}
\pagebreak

\justifying

\pagestyle{fancy}
\fancyhf{}  % Clear default header/footer
\fancyhead[R]{\textcolor{gray}{\nouppercase{\leftmark}}}   % Left header shows current section title
\fancyfoot[C]{\thepage}  % Footer center shows page number

\pagenumbering{roman}

\section*{Abstract}
\markboth{Abstract}{} 
\addcontentsline{toc}{section}{Abstract} 

There is nothing wrong with second place.
Your best effort is all that anyone's asking for.
And if you give your best and you come in second, you come in third, you come in last, it's not about winning or losing.
It's about giving it everything you've got.
Now, Sam has built a monument to devilry and chaos.
I deserve second place, I came in second.
The only crime that's been committed here is that Oscar and Ally deserve first.
We should be applauding them for getting more points.
But in this sick rodeo, this bizarre \textit{fucked up} clown festival that Sam's put together, we're here celebrating what I can only describe as the sickness at the core of America.
And I'm gonna get him, I'm gonna get Sam.

\pagebreak
\section*{Acknowledgements}
\markboth{Acknowledgements}{} 
\addcontentsline{toc}{section}{Acknowledgements}
Thank you to my mother for birthing me and for cutting fruit for me to snack on.

\pagebreak
\renewcommand{\cftdotsep}{0.5}
\renewcommand{\cftsecleader}{\cftdotfill{\cftdotsep}}
\renewcommand{\contentsname}{Table of Contents}  % Set ToC title text
\setlength{\cftbeforesecskip}{10pt}   % Space before sections
\setlength{\cftbeforesubsecskip}{10pt} % Space before subsections
\setlength{\cftbeforesubsubsecskip}{10pt} % Space before subsections
\renewcommand{\cftsecpresnum}{Chapter~} % Adds "Chapter" before section number
\renewcommand{\cftsecaftersnum}{\quad} 
\setlength{\cftsecnumwidth}{6.1em}   %hardcoded in, will die if you have double digit chapters
%\renewcommand{\numberline}[1]{Chapter #1\quad} %old code, works as well but add Chapter to the subsections too which is L
\tableofcontents

\pagebreak
\listoffigures
\addcontentsline{toc}{section}{List of Figures}

\pagebreak
\section*{List of Tables}
\markboth{List of Tables}{} 
\addcontentsline{toc}{section}{List of Tables}

\pagebreak
\section*{List of Appendix Figures}
\markboth{List of Appendix Figures}{} 
\addcontentsline{toc}{section}{List of Appendix Figures}

% \pagebreak
% \section*{Nomenclature}
% \markboth{Nomenclature}{} 
% \addcontentsline{toc}{section}{Nomenclature}

\pagebreak
\pagenumbering{arabic}
\section{Introduction}

\subsection{Motivation for Research}

This research was motivated by the 4AUs it carries (not worth it)

\subsection{Objectives}

\subsection{Scope of Work}

\subsection{Organisation of Report}

\pagebreak
\section{Literature Review}

Ordinarily, when hanging a picture frame, one would simply hang the frame by a singular nail on the wall. 
Obviously, removing that one nail would cause the picture and its frame to fall. 

What if we were to hang the picture by two nails? Hanging the picture in the simplest way, as illustrated in Figure \ref{fig: two_normal}, and removing either nail would not cause the picture frame to fall.
Instead, the picture frame would still hang from the remaining nail.

\begin{figure} %[h]
  \centering
  \includegraphics[width = 0.45\textwidth]{two_normal.png}
  \caption{A normal way to hang a picture on two nails. Adapted from \cite{Demaine_2013}.}
  \label{fig: two_normal}
\end{figure}

If we want to the picture frame to hang on both nails, but fall on the removal of either nail, how would we do so? 
This is the puzzle that was first set forth by A. Spivak in 1997 \cite{Spivak1997StrangePainting}. 
One such solution to the puzzle is as shown in Figure \ref{fig: two_solution}.

\begin{figure}
  \centering
  \includegraphics[width = 0.45\textwidth]{two_solution.png}
  \caption{A solution to the two-nail puzzle. Adapted from \cite{Demaine_2013}.}
  \label{fig: two_solution}
\end{figure}

The puzzle has continued to circulate around the puzzle community, including a spot on Youtuber Tom Scott's channel in collaboration with Jade Tan-Holmes (Up and Atom) \cite{jadetan-holmesupandatomHowKnotHang2019}.
As the puzzle circulates, others have noted certain connections between the solution to the puzzle and other mathematical concepts.

The most notable of which - for this paper at least - being that of Ed Pegg Jr. who mentioned a connection between the solution and Borromean Rings and Brunnian Links, a formalisation that will be discussed further in Section \ref{link formalisation} \cite{Sillke2001StrangePainting}.

First, we will discusss the formalisation using Free Group Theory.

\subsection{Connection to (Free) Group Theory}
This section describes a general framework to study the wrapping of the puzzle as described by Demaine et al. \cite{Demaine_2013} and Ed Pegg Jr. \cite{PictureHanging}. 
We abstract a wrapping of a rope around $n$ nails as a free group on $n$ generators. 
Particularly, we define $2n$ symbols:

\[
x_1, x_1^{-1},x_2, x_2^{-1}, \dots , x_n, x_n^{-1}.
\]

Every $x_i$ symbol represents a wrapping around the $i$th nail; $x_i$ for a clockwise wrapping, and $x_1^{-1}$ for an anti-clockwise one.
This paper will denote the leftmost nail as $x_1$ and the rightmost nail as $x_n$.
Now our wrappings can be represented by some sequence of these $2n$ symbols. 

As an example, the wrapping in Figure \ref{fig: two_solution} can be expressed as $x_1 x_2 x_1^{-1} x_2^{-1}$. 
This is further illustrated by Figure \ref{fig: two_notation}: first we wrap around $x_1$ clockwise, then around $x_2$ clockwise, then anti-clockwise around $x_1^{-1}$, and finally anti-clockwise around $x_2^{-1}$.

\begin{figure}
  \centering
  \includegraphics[width = 0.45\textwidth]{two_notation_clean.jpg}
  \caption{Understanding the notation for Figure \ref{fig: two_solution}. Adapted from \cite{Demaine_2013}.}
  \label{fig: two_notation}
\end{figure}

When removing a nail, for example the $k$th nail, we remove all instances of $x_k$ and $x_k^{-1}$ in a given sequence.
With this representation, it is clear to see why this wrapping is a solution to the 2 nail puzzle. 
For example, removing the leftmost nail ($x_1$) leaves just $x_2 x_2^{-1}$ and removing the rightmost ($x_2$) nail leaves just $x_1 x_1^{-1}$.
The remaining sequence is equivalent to wrapping clockwise then immediately unwrapping the rope with an anti-clockwise turn.
In general, $x_i$ and $x_i^{-1}$ cancel out, thus any instances of $x_i x_i^{-1}$ and $x_i^{-1} x_i$ can be dropped
(The free group specifies that these cancellations are all the simplifications that can be made).

Thus, the original weaving $x_1 x_2 x_1^{-1} x_2^{-1}$ is non-trivially linked with both nails since nothing simplifies;
but removing either nail simplifies the sequence resulting in a trivial wrapping where nothing is linked (i.e. the picture falls).

In group theory, the expression $x_1 x_2 x_1^{-1} x_2^{-1}$ is called the commutator of $x_1$ and $x_2$ and is written [$x_1, x_2$].

As defined in \cite{Demaine_2013}, generally, a picture hanging on $n$ nails to be a sequence of symbols or $word$ in the free group on n generators.
We refer to the $length$ of the word is the number of symbols in the word, this would also approximate the required length of rope to complete the wrapping that corresponds to the given word.
We also define a special identity work (element) 1 that represents the picture falling.
Removing the $i$th nail removes all instances of $x_i$ and $x_i^{-1}$ from the given word.
Through simplification after said removal, we can determine if the picture falls.

\subsection{Connection to Borromean and Brunnian Links}
\label{link formalisation}

\subsection{Research Gaps}

\pagebreak
\section{Research Methodology}
As is the case with pure mathematical research, the research is conducted analytically. 

Firstly, an extensive literature review on the topic of picture-hanging puzzles is conducted to gain pre-requisite knowledge in the field of interest. This review also serves to build an understanding of particularly interesting or impactful open problems that are deserving of greater attention. Having obtained deeper intimacy with the topic, we then apply the theorems and results obtained within the papers reviewed to develop further theories in the field of picture-hanging puzzles that, for example, allow the spectator the most efficient algorithm to remove the fewest nails to fell the picture. We also draw upon the ideas used in proofs to guide our own thoughts. 

In the attempt to solve an open problem, we will put forward various approaches and work through each avenue to see if any of them will crack open the problem.

\subsection{Independent Parameters}

\subsection{Dependent Parameters}

\subsection{Scope of Investigation}

\pagebreak
\section{Model Setup}

\subsection{Heading}

\pagebreak
\section{Results}

\subsection{Overview of Results}

\pagebreak
\section{Conclusions and Recommendations}

\subsection{Conclusions}

\subsection{Recommendations for Future Work}

\pagebreak
\pagenumbering{arabic}
\renewcommand{\thepage}{R-\arabic{page}}

\printbibliography[
  title={References}
]
\addcontentsline{toc}{section}{References}

\pagebreak
\section*{Appendix A - Heading}
\pagenumbering{arabic}
\markboth{Appendix A - Heading}{} 
\renewcommand{\thepage}{A-\arabic{page}}
\addcontentsline{toc}{section}{Appendix A - Heading}

\pagebreak
\section*{Appendix B - Heading}
\pagenumbering{arabic}
\markboth{Appendix B - Heading}{} 
\renewcommand{\thepage}{B-\arabic{page}}
\addcontentsline{toc}{section}{Appendix B - Heading}

\end{document}
